\chapter{Implementation}

\section{OpenCV}
OpenCV is an open source computer vision and machine learning software library. The library has a large amount of optimized algorithms for computer vision. A few areas where OpenCV is used are face recognition, object detection, tracking of moving objects and lane detection. Because OpenCv is a BSD-licensed product, it is free to both utilize and modify the code by companies all over the world. Companies like Google, Microsoft, Intel, Honda and Toyota employ the library for use in various different applications.

OpenCV has C++, C, Python, Java and Matlab interfaces and supports all the major operating systems.

In this project several OpenCV funtions have been utilized mainly in the image processing part. More information about how it is implemented come s in next section.



\section{Raspberry pi}
This section will describe the single board computer that is used for lane detection in this degree project. The chosen board is a Raspberry Pi 3 which is a credit card sized computer. The third generation of the raspberry has seen some major hardware updates compared to earlier versions. The one used in this project has the following specifications:



\begin{table}[H]
\centering
\caption{Raspberry Pi 3 Model B specifications}
\label{my-label}
\begin{tabular}{lllll}
 Model:	&Raspberry Pi 3 Model B  \\
 Operating system:	&Rasbian-Jessie  \\
 Processor:	&ARM Cortex-A53 1.2 GHz 64-bit quad-core  \\
 Hardware Ports:	&40 GPIO pins, 4 USB ports, HDMI port, Ethernet port,\\  &3.5 mm audio jack, Camera interface, Display interface,\\  &Micro SD card slot
\end{tabular}
\end{table}



The main computer in this project is the emc2 board and thus it would be prefered to utilize it for the lane detection as well. But due to no camera drivers available for the emc2 board it would be difficult to manage the image acquisition. A search for other hardware that is more suitable for the task was carried out and the raspberry pi was chosen due to that it is widely used in computer vision projects and its affordable pricepoint.

\subsection{Pi Camera}
The pi camera module has been chosen as image acquisition device for this project. The camera module has a five megapixel image sensor and a maximum resolution of 2952 x 1944 pixels. This camera was chosen due the fact that it is made specifically for the raspberry pi and is very easy to use. The one used has a wideangel lens which maybe is not optimal??





\section{System architecture}


\section{Lane detection algorithm}



This chapter will describe the algorithms used for lane detection that have been implemented on the demonstator of this project. 
